\begin{enumerate}
	\item[(a)] \textit{Differential index of the model? }
	
	Considering no grounding point, we get:
	\begin{align*}
		C\,\left((v_1 - v_2)' - (v_3 - v_2)'\right) &= i;  & e_1\\
		-C\,\left((v_1 - v_2)' - (v_3 - v_2)'\right) + \frac{1}{R}\,(v_3 - v_2) &= 0;  & e_2\\
		v_1 - v_2 &= v(t);  & e_3.
	\end{align*}
	
	Considering ground point 2. By simplifying and differentiating $e_3$, we get:
	\begin{align*}
		v_3' &= \frac{1}{R\,C}\,v_3 + \frac{1}{C}\,v'(t); & e_2\\
		v_1' &= v'(t); & e_3'
	\end{align*}

	\fbox{\begin{minipage}[h!]{\textwidth}
				Therefore, the differential index is 1. 
	\end{minipage}}

	\textit{Does it matter for the index if the grounding point instead had been in point 1 or 3?} 
	
	Considering ground point 2. By simplifying and differentiating $e_3$, we get:
	\begin{align*}
		v_3' &= \frac{1}{R\,C}\,v_3 + \frac{1}{C}\,v'(t); & e_2\\
		v_2' &= -v'(t); & e_3'
	\end{align*}

	\fbox{\begin{minipage}[h!]{\textwidth}
		Therefore, the differential index is 1. 
	\end{minipage}}

	Considering ground point 3. By simplifying and differentiating $e_3$, we get:
	\begin{align*}
		v_1' &= -\frac{1}{R\,C}\,v_1 + \frac{1}{C}\,v'(t); & e_2\\
		v_2' &= -v'(t); & e_3'
	\end{align*}
	
	\fbox{\begin{minipage}[h!]{\textwidth}
			Therefore, the differential index is 1. 
	\end{minipage}}
	The differential index does not change by changing the grounding point.	

	\item[(b)] What is the structural index for the model?
	
	Considering ground point 2, we get:
	\begin{center}
		\begin{tabular}{c|c c c}
			& $v_1'$ & $v_3'$ & i \\
			\hline
			$e_1$ & X & X & X \\
			$e_2$ & X & X &  \\
			$e_3$ &  &  &  \\
		\end{tabular} $\underrightarrow{\text{differentiating}\,e_3}$
	\begin{tabular}{c|c c c}
		& $v_1'$ & $v_3'$ & i \\
		\hline
		$e_1$ & X & X & \textbf{X} \\
		$e_2$ & X & \textbf{X} &  \\
		$e_3'$ & \textbf{X} &  &  \\
	\end{tabular} 
	\end{center}

	\fbox{\begin{minipage}[h!]{\textwidth}
		The matrix now has full rank. The structural index is 2 (1+1). 
	\end{minipage}}
	
	\item[(c)] What is the structural index for the model if the grounding point had been
	in point 1 or 3?
	
	Considering ground point 1, we get:
	\begin{center}
		\begin{tabular}{c|c c c}
			& $v_2'$ & $v_3'$ & i \\
			\hline
			$e_1$ & & X & X \\
			$e_2$ & & X &  \\
			$e_3$ & & &  \\
		\end{tabular} $\underrightarrow{\text{differentiating}\,e_3}$
	\begin{tabular}{c|c c c}
		& $v_2'$ & $v_3'$ & i \\
		\hline
		$e_1$ & & X & \textbf{X} \\
		$e_2$ & & \textbf{X} &  \\
		$e_3'$ & \textbf{X} & &  \\
	\end{tabular}
	\end{center}
	
	\fbox{\begin{minipage}[h!]{\textwidth}
			The matrix now has full rank. The structural index is 2 (1+1). 
	\end{minipage}}
	
	Considering ground point 3, we get:
	\begin{center}
		\begin{tabular}{c|c c c}
			& $v_1'$ & $v_2$ & i \\
			\hline
			$e_1$ & X &  & \textbf{X} \\
			$e_2$ & X & \textbf{X} &  \\
			$e_3$ & \textbf{X} & X &  \\
		\end{tabular} 
	\end{center}
	
	\fbox{\begin{minipage}[h!]{\textwidth}
			The matrix now has full rank. The structural index is 1 (0+1). 
	\end{minipage}}
	
	\item[(d)]  Index and the structural index for the model
	\textit{Differential index}
	\begin{align*}
		\dot x + \dot y + x + y &= \cos(t) & e_1\\
		\dot x + \dot y + x + 2\,y &= t & e_2\\		
	\end{align*}
	Consider $e_3$ = $e_2 - e_1$, i.e:
	\begin{align*}
		y &= t - \cos(t) & e_3\\
	\end{align*}
	differentiating $e_3$, we get:
	\begin{align*}
		\dot x + \dot y + x + y &= \cos(t) & e_1\\
		\dot x + \dot y + x + 2\,y &= t & e_2\\	
		\dot y &= 1 + \sin(t) & \dot e_3\\
	\end{align*}	
	\fbox{\begin{minipage}[h!]{\textwidth}
		The differential index is 1
	\end{minipage}}	
	
	\textit{Structural index}\\
	Considering $e_1$ and $e_2$, we get:
	\begin{center}
		\begin{tabular}{c|c c}
			& $\dot x$ & $\dot y$ \\
			\hline
			$e_1$ & X & X \\
			$e_2$ & X & X 
		\end{tabular}  Full rank? No! The matrix is actually: \begin{tabular}{c|c c}
			& $\dot x$ & $\dot y$ \\
			\hline
			$e_1$ & 1 & 1 \\
			$e_2$ & 1 & 1 
		\end{tabular} 
	\end{center}	
	\fbox{\begin{minipage}[h!]{\textwidth}
		Considering just $e_1$ and $e_2$, we get the structural index as 0.
	\end{minipage}}	
	
	Considering $e_1$, and $e_3$, we get:
	\begin{center}
		\begin{tabular}{c|c c}
			& $\dot x$ & $\dot y$ \\
			\hline
			$e_1$ & X & X \\
			$e_3$ &  &  
		\end{tabular}  $\underrightarrow{\text{differentiating}\,e_3}$ 
		\begin{tabular}{c|c c}
			& $\dot x$ & $\dot y$ \\
			\hline
			$e_1$ & \textbf{X} & X \\
			$\dot e_3$ &  & \textbf{X} 
		\end{tabular} 
	\end{center}	
	\fbox{\begin{minipage}[h!]{\textwidth}
			Considering just $e_1$ and $e_3$, we get the structural index as 1.
	\end{minipage}}	
\end{enumerate}