\section{PMSM Model}
This section presents the model of a PMSM electric machine. 

Instantaneous three-phase voltages (\va, \vb, and \vc) and currents (\ia, \ib , and \ic) can be written as follows:
\begin{align*}
	\va(t) &= \hat\va\,\cos\left(\we\,t\right), & \vb(t) &= \hat\vb\,\cos\left(\we\,t + \frac{2\,\pi}{3}\right), & \vc(t) &= \hat\vc\,\cos\left(\we\,t - \frac{2\,\pi}{3}\right), \\
	\ia(t) &= \hat\ia\,\cos\left(\we\,t - \phi\right), & \ib(t) &= \hat\ib\,\cos\left(\we\,t + \frac{2\,\pi}{3} - \phi\right), & \ic(t) &= \hat\ic\,\cos\left(\we\,t - \frac{2\,\pi}{3} - \phi\right).
\end{align*}
Assuming a balanced 3-phase network, i.e:
\begin{align*}
	\va + \vb + \vc &= 0, & \ia + \ib + \ic &=0
\end{align*}
Transformation to an equivalent two-phase system, i.e, Clarke transformation, is presented as follows: 
\begin{align*}
	\begin{pmatrix}
		\valpha \\
		\vbeta
	\end{pmatrix} &= \frac{2\,K}{3} \begin{pmatrix}
										1 & -\frac{1}{2} & -\frac{1}{2} \\
										0 & \frac{\sqrt3}{2} & -\frac{\sqrt{3}}{2} \\
									\end{pmatrix}\, \begin{pmatrix}
														\va \\
														\vb \\
														\vc
													\end{pmatrix}, & 	\begin{pmatrix}
													\ialpha \\
													\ibeta
												\end{pmatrix} &= \frac{2\,K}{3} \begin{pmatrix}
												1 & -\frac{1}{2} & -\frac{1}{2} \\
												0 & \frac{\sqrt3}{2} & -\frac{\sqrt{3}}{2} \\
											\end{pmatrix}\, \begin{pmatrix}
											\ia \\
											\ib \\
											\ic
										\end{pmatrix},
\end{align*}
where $K$ is a constant (1, peak-scaling and $\sqrt{3/2}$ for power-scaling).

In-order to simplify the analysis, the reference frame of a three-phase vectors are rotated. i.e:
\begin{align*}
	\begin{pmatrix}
			\vd \\
			\vq
	\end{pmatrix} &= \begin{pmatrix}
						\cos(\we\,t) & -\sin(\we\,t) \\
						\sin(\we\,t) & \cos(\we\,t) \\
					 \end{pmatrix}\, \begin{pmatrix}
										 \valpha \\
										 \vbeta
									 \end{pmatrix} & \begin{pmatrix}
									 \id \\
									 \iq
								 \end{pmatrix} &= \begin{pmatrix}
								 \cos(\we\,t) & -\sin(\we\,t) \\
								 \sin(\we\,t) & \cos(\we\,t) \\
							 \end{pmatrix}\, \begin{pmatrix}
							 \ialpha \\
							 \ibeta
						 \end{pmatrix}
\end{align*}
The PMSM can thus be modeled as follows:
\begin{align*}
	\Ld\,\frac{d\id}{dt} &= \vd - \Rs\,\id + \Lq\,\iq\,\wm,\\
	\Lq\,\frac{d\iq}{dt} &= \vq - \Rs\,\iq - \left(\Ld\,\id + \flux\right)\,\wm,\\
	\J\,\frac{d\wm}{dt} &= \frac{3\,\np}{2\,K^2}\,\left(\left(\Ld - \Lq\right)\,\id\,\iq + \iq\,\flux\right) - \Tl,
\end{align*}
where \wm is PMSM rotor rotating magnetic field's angular velocity, \Rs is the stater resistance per phase, \Ld and \Lq are $d$ and $q$ axis inductances, respectively, \flux is the flux constant of the permanent magnet, \np is the number of pole pairs, and \Tl is the load torque. 

% \printinunitsof{mm}\prntlen{\textwidth} (\printinunitsof{in}\prntlen{\textwidth})
% \printinunitsof{mm}\prntlen{\textheight} (\printinunitsof{in}\prntlen{\textheight})

\fbox{\begin{minipage}[h!]{\textwidth}
	The system equations are given below:
	\begin{align}
		\va &= \hat\va\,\cos(\we\,t), & (\text{input}),\\
		\vb &= \hat\vb\,\cos(\we\,t + \frac{2\,\pi}{3}), & (\text{input}),\\
		\vc &= \hat\vc\,\cos(\we\,t - \frac{2\,\pi}{3}), & (\text{input}),\\
	%% --- alpha beta
		\valpha &= \frac{2\,K}{3}\,\left(\va -\frac{1}{2}\,\vb -\frac{1}{2}\,\vc\right), \\
		\vbeta &= \frac{2\,K}{3}\,\left(\frac{\sqrt3}{2}\,\vb -\frac{\sqrt3}{2}\,\vc\right), \\
	%% --- dq
		\vd &= \cos(\we\,t)\,\valpha - \sin(\we\,t)\,\vbeta,\\
		\vq &= \sin(\we\,t)\,\valpha + \cos(\we\,t)\,\vbeta,\\ 
%		\begin{pmatrix}
%			\vd \\
%			\vq
%		\end{pmatrix} &= \begin{pmatrix}
%			\cos(\we\,t) & -\sin(\we\,t) \\
%			\sin(\we\,t) & \cos(\we\,t) \\
%		\end{pmatrix}\, \begin{pmatrix}
%			\valpha \\
%			\vbeta
%		\end{pmatrix} \\
	%% --- state equations
		\Ld\,\frac{d\id}{dt} &= \vd - \Rs\,\id + \Lq\,\iq\,\wm,\\
		\Lq\,\frac{d\iq}{dt} &= \vq - \Rs\,\iq - \left(\Ld\,\id + \flux\right)\,\wm,\\
		\J\,\frac{d\wm}{dt} &= \frac{3\,\np}{2\,K^2}\,\left(\left(\Ld - \Lq\right)\,\id\,\iq + \iq\,\flux\right) - \Tl,\\
	%% --- output variables
		\is &= \id + j\,\iq, & (\text{output}),\\
		\Te &= \frac{3\,\np}{2\,K^2}\,\left(\left(\Ld - \Lq\right)\,\id\,\iq + \iq\,\flux\right), & (\text{output}),\\
		\Nr &= \frac{\wm}{\np}\,\frac{30}{\pi}, & (\text{output}).
	\end{align}
\end{minipage}}

The above gives the ODE with three dynamic variables, \id, \iq, \wm. 
	
The main aim of the project is to show that the numerical solution gives the same solution as the differential equation. However, we cannot determine the solution analytically. Therefore, we need to try perform the several checks. The following sections cover these in detail. 

Two different single step numerical methods are chosen: a fixed-step and a variables step length methods. \textit{Euler forward} method is used as the fixed step solver. The \textit{Dormand-Prince 4(5)} method is employed as the variable step length solver.

The table \prettyref{tab:simparam} below provides the simulation parameters:
\begin{table}[h!b!]
	\centering
	\begin{tabular}{c|c|c}
		Parameter Name & Parameter symbol & value \\
		\hline
		Peak input voltage & $\hat \va,\ \hat \vb,\ \hat \vc $ & 800\,V\\
		Load torque & \Tl & 0\\
		Motor inertia & \J & 1.09\,kgm$^2$\\
		Number of pole pairs & \np & 20\\
		Rated speed (Nominal speed) & \Nrnom & 1000\,rpm \\
		Operating speed & \Nr & 960\,rpm \\
		Electrical frequency & \textit{f\textsubscript{1}} & 166.67\,Hz\\
		Stator inductance (direct axis)& \Ld & 48.35\,$\mu$H\\
		Stator inductance (quadrature axis)& \Ld & 48.35\,$\mu$H\\
		Stator resistance & \Rs & 0.052 \\
		Flux constant & $\Psi$ & 0.38 \\
		Clarke Constant & $K$ & 1
	\end{tabular}
	\caption{Simulation parameters for the PMSM}
	\label{tab:simparam}
\end{table}
